\documentclass{beamer}
\usepackage[noend]{algpseudocode} 

\usepackage[T1,T2A]{fontenc}
\usepackage[utf8]{inputenc}
\usepackage[russian]{babel}
\usepackage{tempora}
\setbeamertemplate{navigation symbols}{}

\usetheme{Hannover}
\usecolortheme{lily}


\defbeamertemplate*{title page}{customized}[1][]
{
	{\center
		\usebeamerfont{institute}\insertinstitute\par
		\bigskip
		\usebeamerfont{title}\inserttitle\par
		\usebeamerfont{subtitle}\insertsubtitle\par
		\par}
	\bigskip
	\bigskip
	{\flushright
		\usebeamerfont{author}\insertauthor\par
		\par}
	{\center
		\vspace{\fill}
		\usebeamerfont{date}\insertdate\par
		\par}
}

\hypersetup{
	unicode=true,
}
\title{Матрицы и группы}
\subtitle{Введение в специальность}  
\institute{БФУ Имени И.Канта ИФМНиИТ}
\author{Флягин Артём Иванович \\
	Компьютерная безопасность I}

\date{1 июля 2020}

\begin{document}
	\frame{\titlepage} 
	\begin{frame}{План лекции}
		\begin{enumerate}
			\item Матрицы
			\item Матричные криптосистемы
			\item Группы
			\item Подргуппы
			\item Практика
		\end{enumerate}
	\end{frame}
	\begin{frame}{Основные понятия}
		\begin{enumerate}
			\item Матрицы
			\item Алгебра
			\item Определитель матрицы
			\item Обратимая матрица
			\item Группа
			\item Гомоморфизм
			\item Изоморфизм
			\item Подгруппа
		\end{enumerate}
	\end{frame}
	\begin{frame}{Цели}
		\begin{enumerate}
			\item Вспомнить основные понятия о матрицах из курса алгебры
			\item Познакомиться с матричными криптосистемами
			\item Более углубленно ознакомиться с понятиями грппы и подгруппы, систиматизировать определения.
		\end{enumerate}
	\end{frame}
\end{document}